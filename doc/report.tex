\documentclass[11pt,a4paper,notitlepage]{article}

% PDF name of document
\newcommand{\doctitlepdf}{Notes on PFOO simple powerflow solver}
% Title of document
\newcommand{\doctitle}{\doctitlepdf}
% Author of document
\newcommand{\docauthor}{Joel Goop}
\newcommand{\docauthorpdf}{\docauthor}

% Packages
\usepackage[utf8]{inputenc}
\usepackage[T1]{fontenc}
\usepackage[english]{babel}
\usepackage{amsmath}
\usepackage{amssymb}
\usepackage{mathtools}
\usepackage{ae}
\usepackage{units}
\usepackage{listings}
\usepackage{xcolor}
\usepackage[pdftex]{graphicx}
\usepackage{epstopdf}
\usepackage{caption}
\usepackage[subrefformat=parens,labelformat=parens]{subfig}
\usepackage[round]{natbib}
\usepackage{multirow}
\usepackage{array}
\usepackage{geometry}
\usepackage{fancyhdr}
\usepackage[hyphens]{url}
\usepackage[breaklinks,pdfpagelabels=false]{hyperref}
\usepackage{enumitem}
\usepackage{textcomp}
%\usepackage[utopia]{mathdesign}


% New commands
\newcommand{\setdefaulthdr}{%
\fancyhead[L]{\docauthorpdf}%
\fancyhead[R]{\doctitlepdf}%
\fancyfoot[C]{\thepage}%
}
\newcommand{\setspecialhdr}{%
\fancyhead[L]{ }%
\fancyhead[R]{\slshape \leftmark}%
\fancyfoot[C]{\thepage}%
}
\newcommand{\mail}[1]{\href{mailto:#1}{\nolinkurl{#1}}}
\newcommand{\backgroundpic}[3]{%
	\put(#1,#2){
		\parbox[b][\paperheight]{\paperwidth}{%
			\vfill
			\centering
			\includegraphics[width=\paperwidth,height=\paperheight,keepaspectratio]{#3}
			\vfill
}}}

\renewenvironment{abstract}%
{\begin{center} \bfseries \abstractname \end{center}}%
{\vspace{2\baselineskip}}%
\newcommand{\sectionun}[1]{\section*{#1}\addcontentsline{toc}{section}{\numberline{}#1}}
\newcommand{\subsectionun}[1]{\subsection*{#1}\addcontentsline{toc}{subsection}{\numberline{}#1}}
\newcommand{\subsubsectionun}[1]{\subsubsection*{#1}\addcontentsline{toc}{subsubsection}{\numberline{}#1}}
\DeclareMathOperator*{\Var}{Var}

% Configuration
\captionsetup{margin=10pt,font=small,labelfont=bf}
\captionsetup[table]{position=top}
\lstset{basicstyle=\ttfamily,breaklines=true}
\setlength{\extrarowheight}{4pt}
\addtolength{\headheight}{\baselineskip}
\pagestyle{fancy}
%\setdefaulthdr
\bibliographystyle{plainnat}
\hypersetup{
	pdftitle={\doctitlepdf},%
	pdfauthor={\docauthorpdf},%
    colorlinks=true,%
    citecolor=black,%
    filecolor=black,%
    linkcolor=black,%
    urlcolor=black
}
\setenumerate{listparindent=\parindent,parsep=\parskip}

% Alter some LaTeX defaults for better treatment of figures:
    % See p.105 of "TeX Unbound" for suggested values.
    % See pp. 199-200 of Lamport's "LaTeX" book for details.
    %   General parameters, for ALL pages:
    \renewcommand{\topfraction}{0.9}	% max fraction of floats at top
    \renewcommand{\bottomfraction}{0.8}	% max fraction of floats at bottom
    %   Parameters for TEXT pages (not float pages):
    \setcounter{topnumber}{2}
    \setcounter{bottomnumber}{2}
    \setcounter{totalnumber}{4}     % 2 may work better
    \setcounter{dbltopnumber}{2}    % for 2-column pages
    \renewcommand{\dbltopfraction}{0.9}	% fit big float above 2-col. text
    \renewcommand{\textfraction}{0.07}	% allow minimal text w. figs
    %   Parameters for FLOAT pages (not text pages):
    \renewcommand{\floatpagefraction}{0.7}	% require fuller float pages
	% N.B.: floatpagefraction MUST be less than topfraction !!
    \renewcommand{\dblfloatpagefraction}{0.7}	% require fuller float pages

	% remember to use [htp] or [htpb] for placement
\definecolor{listinggray}{gray}{0.9}
\definecolor{lbcolor}{rgb}{0.9,0.9,0.9}
\lstset{
	backgroundcolor=\color{lbcolor},
	language={},
	tabsize=4,
	rulecolor=,
	basicstyle=\ttfamily\scriptsize,
	upquote=true,
	columns=fixed,
	showstringspaces=false,
	extendedchars=true,
	breaklines=true,
	prebreak = \raisebox{0ex}[0ex][0ex]{\ensuremath{\hookleftarrow}},
	frame=single,
	showtabs=false,
	showspaces=false,
	showstringspaces=false,
	keywordstyle=\color[rgb]{0,0,1},
	commentstyle=\color[rgb]{0.133,0.545,0.133},
	stringstyle=\color[rgb]{0.627,0.126,0.941},
}
\renewcommand{\vec}[1]{\ensuremath{\boldsymbol{#1}}}

\begin{document}
\selectlanguage{english}

\pagenumbering{alph}
\setcounter{page}{1}

% Titelsida och abstract
\title{\doctitle}
\author{\docauthor}
\date{\today}
\maketitle

% Huvuddel
\setdefaulthdr
\pagenumbering{arabic}
\setcounter{page}{1}

\section{Program description} % (fold)
\label{sec:program_description}
The \texttt{Python} program contains a set of classes designed to describe and solve power flow problems. The classes are contained in the file \texttt{pf\_classes.py}. The class \texttt{PowerSystem} describes the buses, lines, and transformers making up a power system. Transformers and lines are both described by the class \texttt{Line}, which contains info on series impedance, $R+iX$, and shunt susceptance, $B$, equally divided between the two ends of the line. The class \texttt{Bus} is designed to represent buses in the system. The attributes represented are voltage amplitude $|V|$, voltage angle $\delta$, generated real and reactive power $P_g$ and $Q_g$, real and reactive load $P_l$ and $Q_l$, minimum and maximum generated reactive power (currently not used in calculations) $Q_{g,\text{min}}$ and $Q_{g,\text{max}}$, as well as information on whether the bus is a slack bus, a voltage controlled bus, or neither.

In order to solve the power flow problem described by an instance of the \texttt{PowerSystem} class, different solution methods are implemented in the subclasses of \texttt{PowerFlowSolver}. The currently implemented methods are: full Newton method (\texttt{Power\-Flow\-Full\-Newton\-Solver}), fast decoupled power flow (\texttt{Power\-Flow\-Fast\-Decoupled\-Solver}), and DC power flow (\texttt{Power\-FlowDCSolver}).
% section program_description (end)

\section{Usage} % (fold)
\label{sec:usage}
An example of usage can be found in the file \texttt{gso\_ex\_6\_9.py} (see Appendix~\ref{app:code}) which solves example 6.9 of \citep{glo2012}. To describe the system, lines, transformers, and buses have to be described, and a \texttt{PowerSystem} object must be created. Then a solver object from a selected solver class can be created (for some solvers, e.g. the full Newton solver, additional parameters such as tolerance and maximum number of iterations should be specified) and its \texttt{solve()} method can then be called.
% section usage (end)

% \begin{figure}[htp]%
% 	\centering%
% 	\includegraphics[width=0.95\textwidth]{fig/num_regs}%
% 	\caption[]{Number of regions in solution with $\omega_i>0.0001$ as a function of $\alpha$, $\beta$, and $\gamma$.}%
% 	\label{fig:num_regs}%
% \end{figure}

% \begin{figure}[htp]%
% 	\centering%
% 	\subfloat[Frontier curves obtained for different fixed values of $\alpha$, representing the weight of the penalty on high variance.]{%
% 		\includegraphics[width=0.85\textwidth]{fig/frontiers_distance-output}%
% 		\label{fig:fr_dist-out}%
% 	}\\
% 	\subfloat[Frontier curves obtained for different fixed values of $\beta$, representing the weight of the penalty on low output.]{%
% 		\includegraphics[width=0.85\textwidth]{fig/frontiers_distance-stdev}%
% 		\label{fig:fr_dist-std}%
% 	}\\
% 	\subfloat[Frontier curves obtained for different fixed values of $\gamma$, representing the weight of the penalty on large distances.]{%
% 		\includegraphics[width=0.85\textwidth]{fig/frontiers_stdev-output}%
% 		\label{fig:fr_std-out}%
% 	}%
% 	\caption[]{Figures representing the efficient frontier by showing the trade-offs between two objectives, by varying two of the weight parameters, for different fixed values of the third parameter. $\alpha$ is fixed in \subref{fig:fr_dist-out}, $\beta$ is fixed in \subref{fig:fr_dist-std}, and $\gamma$ is fixed in \subref{fig:fr_std-out}.}%
% 	\label{fig:frontiers}%
% \end{figure}

\clearpage
\newpage
\addcontentsline{toc}{chapter}{\numberline{}References}
\bibliography{references}

\clearpage
\newpage
\appendix
\renewcommand{\lstlistingname}{File}
\lstset{
    language=python,
        caption=\lstname,
}
\section{Code}\label{app:code}
\lstinputlisting{../pf_classes.py}
\lstinputlisting{../gso_ex_6_9.py}

\end{document}
% PDF author of document
